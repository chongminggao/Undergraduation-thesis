% !Mode:: "TeX:UTF-8"

\chapter{绪论}
\section{引言-从数据挖掘谈起}
% 计算电磁学方法\citeup{wang1999sanwei,
% liuxf2006,
% zhu1973wulixue,
% chen2001hao,
% gu2012lao,
% feng997he}
% 从时、频域角度划分可以分为频域方法与时域方法两大类。
% 频域方法的研究开展较早,目前应用广泛的包括:矩量法(MOM)\citeup{xiao2012yi,zhong1994zhong}及其快速算
% 法多层快速多极子(MLFMA)\citeup{clerc2010discrete}方法、有限元(FEM)\citeup{wang1999sanwei,zhu1973wulixue}方法、自适应积分(AIM)
% \citeup{gu2012lao}方法等,这些方法是目前计算电磁学商用软件
% \footnote{脚注序号“①,……,⑩”的字体是“正文”,不是“上标”,序号与脚注内容文字之间空1个半角字符,脚注的段落格式为:单倍行距,段前空0磅,段后空0磅,悬挂缩进1.5字符;中文用宋体,字号为小五号,英文和数字用Times New Roman字体,字号为9磅;中英文混排时,所有标点符号(例如逗号“,”、括号“()”等)一律使用中文输入状态下的标点符号,但小数点采用英文状态下的样式“.”。}
% (例如:FEKO、Ansys 等)的
% 核心算法。由文献\cite{feng997he,clerc2010discrete,xiao2012yi}可知……
现如今,我们处于一个充满数据的时代。在每一天我们使用计算机、手机时候,都有大量数据产生,接着被以各种形式记录、保留下来。许多数据都给我们的生活提供着极大的便利,比如根据用户的音乐的历史记录,音乐软件能推荐更多的适合该客户口味的音乐;再比如当我用键盘输入这段文字的时候,中文输入软件根据词库里的数据,将键入的字母序列转换为一段段可能性最大的中文词汇或句子。在经济学、医学、生物学、社会科学等学科中,海量的数据无时无刻不在产生,然而如何合理地利用这些数据,使其提供我们需要的信息,成为了现在各个领域面临的问题。

在这样多领域的需求下,数据挖掘(Data Mining,缩写:DM)这门交叉学科应运而生。通常来说,数据挖掘是数据库知识发现(Knowledge-Discovery in Databases,缩写:KDD)中的一个步骤,其目的是在大量的数据中自动搜索隐藏于其中的特殊信息,从而为之后的分析决策提供理论依据。下面将简要介绍下数据挖掘的主要步骤:
\vspace{4mm}
\pic[h]{数据挖掘主要步骤图}{}{overview}
\begin{itemize}
    \item \textbf{~~数据采集} 所有工作开始之前,首先需要采集数据,包括确定数据种类、范围等,然后对数据进行初步选择,挑选出合适的数据。
    \item \textbf{~~数据预处理} 该过程包括对原始数据的处理,包括数据整合、去除噪声等。
    \item \textbf{~~数据转化} 对数据进行完预处理后,需要决定数据合适表示,例如特征选筛等。
    \item \textbf{~~数据挖掘} 这个过程中,人们采用各种方法,例如聚类、分类、关联规则分析等方法来发掘数据中的有用的信息。
    \item \textbf{~~结果评估与可视化} 最后,需要对得到的结果进行解释与评估,并可视化为易于人理解的形式,在这之后有可能需要重新进行挖掘。
\end{itemize}

\vspace{2mm}
这其中,\textbf{数据挖掘}是从数据中学习知识的最关键的步骤,因此很多时候,数据挖掘泛指从数据中学习知识的过程。数据挖掘的大量算法可以按照目的分为以下四类:

\begin{itemize}
    \item \textbf{~~分类(Classification)} 分类算法的目的是为特定变量确定类别或者标签,比如根据近年来我国的经济发展情况来确定房价是涨还是跌。一般来说,分类首先用历史数据作为训练集,学习出目标函数,然后用学到的目标函数来预测新来的未知数据点的类别。常见的分类算法有\emph{kNN}\cite{peterson2009k},\emph{决策树}\cite{quinlan1986induction},\emph{支持向量机}\cite{cortes1995support}等。
    \item \textbf{~~聚类(Clustering)} 聚类算法的目的是将数据分为许多类,使得相似的数据分在同一类中,不相似的数据分布在不同的类中,比如菜农可以根据一批辣椒的形状、辛辣程度将其聚拢成不同类别销售。常见的聚类方法有\emph{k-means}\cite{hartigan1979algorithm}, \emph{spectral clustering}\cite{ng2002spectral} 等方法。
    \item \textbf{~~关联规则分析(Analysis of Association Rule)} 关联规则分析的目的是从数据中发现经常出现的模式,一个经典的例子是人们从超市的大量销售记录中发现买尿布的人也常常买啤酒。经典的关联规则分析方法有:\emph{Apriori}\cite{agrawal1994fast}, \emph{DBSCAN}\cite{ester1996density}和\emph{FP-growth}\cite{han2000mining}等。
    \item \textbf{~~奇异点检测(Animation Detection)} 奇异点检测的目的是发现数据集中存在的奇异点,即与大多数点不相似的少数数据点,比如邮件代理公司会根据正常邮件与垃圾邮件的特征对比,来为用户标记垃圾邮件。通常来说大多数聚类算法都可以作为奇异点检测算法。
\end{itemize}

\vspace{2mm}
相对于数据挖掘的其他算法,聚类的知识目前还不够系统化。一个重要原因是聚类不存在客观标准:给定数据集,总能从某个角度找到以往算法未覆盖的某种标准从而设计出新算法\cite{estivill2002so}。但聚类技术本身在现实任务中非常重要,近些年关于聚类的新算法在数据挖掘、机器学习、人工智能的顶级会议乃至《自然》和《科学》上都频出不穷。本文也将提出一种全新的基础聚类算法,在此之前,先引入由特殊需求引入的新型聚类技术:双边聚类技术。



\section{双边聚类技术}
聚类技术有很多变种,其中双边聚类(Co-Clustering,或Bi-Clustering,Two-mode clustering)就是一种,其致力于突破传统聚类的限制,在两个空间中同时进行聚类,从而创造更好的应用价值。现在首先来谈谈双边聚类是什么,随之引入一个正式的双边聚类的问题描述。

\subsection{“同时聚类”需求的产生}
在传统的聚类中,对于一个数据集,总是给定一个特征空间,对数据集进行聚类。比如在传统的“用户-商品”推荐系统(Recommondaton System)中,想对用户进行聚类,那就要将各种商品作为特征集,通过不同用户喜欢的商品集合的异同来判断用户之间的相似性,从而最终达到对用户聚类的目的。同样的,如果想对商品集合进行聚类,那反之得将商品集合作为数据集,用户作为特征集,对商品进行聚类。至于聚类的目的,对同一个用户簇可以推荐相同的商品,而同一个商品簇可以归类到一起进行管理,这是后话了。

同样的,对于文本挖掘(Text Mining),大量的词汇和文档也可以组成两个空间,将其中一个空间作为特征集,对另外一个进行聚类,此类例子还有很多。那么,有的时候,我们不禁发问:能否同时对两个空间进行聚类?比如在推荐系统中同时对用户集合和商品集合进行聚类,于是在得到相似的用户的群组的同时,得到该群组喜欢的商品集合!同样地,在文本挖掘中我们是否可以在得到相似的文本集的同时,得到该文本集包含的词汇集?

要是这种两个空间“同时聚类”的问题能够解决,那么在大量应用中将得到极好的结果和可解释性,甚至颠覆传统聚类方法的价值,从而为科研和生产提供全新的方向。事实上,现在已经有大量的双边聚类算法诞生并且投入应用,先来看看最初的双边聚类算法是怎么出现的。

\subsection{双边聚类问题描述}
双边聚类问题诞生于生物信息学中的基因表达问题(Gene Expression Profiling)上,简单来说,近年来生物信息检测技术的进步为科学家们提供了大量的基因表达数据,关于,可以用一个“基因-样本”的矩阵来表示。

依据传统的聚类技术,我们可以将基因进行聚类,如\ref{gene-condition}所示;也可以将样本

\pic[htb]{数据挖掘主要步骤图}{width=150mm}{gene-condition}


\section{本文的主要贡献与创新}
本论文以时域积分方程时间步进算法的数值实现技术、后时稳定性问题以及
两层平面波加速算法为重点研究内容,主要创新点与贡献如下:

……
\section{本论文的结构安排}
本文的章节结构安排如下:

……
