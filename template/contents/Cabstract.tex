% !Mode:: "TeX:UTF-8"

\begin{Cabstract}{双边聚类}{同步}{基因表达数据}{}{}
近年来,双边聚类成为了生物信息学、文本挖掘和推荐系统里广受关注的问题,越来越多的研究工作围绕其展开。目前存在的相关算法大多以启发式搜索的方式来找寻数据矩阵中的联合簇,搜索指标的选取决定了这类算法的性能。在本文中,我们提出了一种全新的基于同步思想的双边聚类算法CoSync,其工作方式为用动态交互的方式来自动第发掘嵌入数据矩阵中的联合簇。CoSync算法的核心思想是将数据矩阵视为一个动态系统,矩阵中每一个元素的值都将受到其邻居元素值的加权影响,随着迭代属于同一联合簇子矩阵的元素值终将同步到一起,构成一个常数值子块。我们将展示CoSync吸引人的优势:(a) CoSync可以根据数据内在的结果,自动识别出高质量的联合簇。(b) CoSync对联合簇在数据矩阵中的位置分布没有限制,其可以发掘现实数据中分布复杂的联合簇。(c) 加入非负矩阵分解的模块后,CoSync可以对任意高维数据矩阵进行分析。最后,实验证明我们的算法能够成功发掘出数据集中高质量的联合簇,且性能超过国际上有代表性的其他算法。
\end{Cabstract}

% \newpage\mbox{}\thispagestyle{empty}\newpage
% \addtocounter{page}{-1}
