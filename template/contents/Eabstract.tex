% !Mode:: "TeX:UTF-8"

\addtocounter{page}{-1}
\begin{Eabstract}{Co-clustering}{Synchronization}{Gene Expression Data}{}{}
Co-clustering has gained growing attentions recently due to its wide practical applications in biological data analysis, text mining and recommender systems. Most existing co-clustering algorithms usually search co-clusters by heuristic searching algorithm, the performance depends on the choosing of criteria. In this paper, we propose a new synchronization-inspired co-clustering algorithm by dynamic simulation, called CoSync, which aims at discovering all co-clusters embedding in a given gene expression data matrix. The basic idea is to view the data matrix as a dynamical system, and the weighted two-sided interactions are imposed on each entry of the matrix from both aspects of rows and columns,  resulting in the values of all entries in a co-cluster synchronizing together. We demonstrate that our new co-clustering approach has several attractive benefits: (a) CoSync is capable of identifying co-clusters with high-quality, driven by the intrinsic data structure. (b) Without any co-cluster structure assumption, CoSync supports finding co-clusters of arbitrary size, not limited to disjoint co-clusters.  (c)  In conjunction with non-negative matrix factorization, CoSync allows analyzing large-scale data. Experiments show that our algorithm faithfully uncovers co-clusters embedded in data sets and has good performance compared to state-of-the-art algorithms.
\end{Eabstract}

\newpage\mbox{}\thispagestyle{empty}\newpage
\addtocounter{page}{-1}
